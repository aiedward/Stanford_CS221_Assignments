\documentclass[12pt]{article}
\usepackage{fullpage,enumitem,amsmath,amssymb,graphicx}

\begin{document}

\begin{center}
{\Large CS221 Fall 2015 Homework [number]}

\begin{tabular}{rl}
SUNet ID: & ziyuanc \\
Name: & Ziyuan Chen \\
Collaborators: & N/A
\end{tabular}
\end{center}

By turning in this assignment, I agree by the Stanford honor code and declare
that all of this is my own work.

\section*{Problem 1}

\begin{enumerate}[label=(\alph*)]
  \item $\frac{df(x)}{dx}=\sum_{i=1}^{n} (\omega_ix-\omega_ib_i)=0$\newline
  $x=\frac{\sum_{i=1}^{n} \omega_1b_i}{\sum_{i=1}^{n} \omega_i}$ minimizes the function.\newline
  If some $\omega_i$'s are negative, then second derivative of $f(x)$ could be negative. Therefore, the value $x$ obtained from the expression above could be the maximum instead of the minimum.
  \item We need to refactor the expressions\newline
  $f(x) = \max_{a \in \{1,-1\}} a \sum_{j=1}^d x_j=|\sum_{j=1}^d x_j|$\newline
  $g(x) = \sum_{j=1}^d \max_{a \in \{1,-1\}} a x_j=\sum_{j=1}^d |x_j|$\newline
  According to the triangle inequality, $g(x)\geqslant f(x)$
  \item Use the geometric distribution\newline
  The expected number of rolls before stopping is $\frac{1}{\frac{1}{3}}=3$. For each roll, the chance of getting $r$ points is $\frac{1}{6}$. Thus the expected number of points is $3\times \frac{1}{6}r=\frac{1}{2}r$
  \item Take the derivative of $logL(p)$:\newline
  $\frac{dlogL(p)}{dp}=\frac{d(2log(p)+3log(1-p))}{dp}=0$\newline
  So $\frac{2}{p}=\frac{3}{1-p}=\frac{2}{p}-\frac{3}{1-p}=0$
  $p=\frac{2}{5}$\newline
  If we take the second derivative, then we have 
  $\frac{-2}{p^2}-\frac{3}{(1-p)^2}$ which is $-20.8$ at the $p$ we computed above. Thus it is a maximum point.
  \item $\nabla f(\mathbf w) = 2\sum_{i=1}^n \sum_{j=1}^n (a_i^\top - b_j^\top) (a_i^\top \mathbf w - b_j^\top \mathbf w) + 2\lambda \mathbf w,$
\end{enumerate}

\section*{Problem 2}

\begin{enumerate}[label=(\alph*)]
  \item A rectangle has (n-a+1)(n-b+1) placements if it is a$\times$b. Therefore the number of possible placements of a rectangle is $O(n^2).$\newline
  Because both a and b are in the range [1,n], a rectangle has $O(n^2)$ different possible dimensions.\newline
  Therefore for one rectangle there are $O(n^4)$ ways to place it.\newline
  For 5 rectangles, it should be in the order of $(n^4)^5=n^{20}$. Thus it's $O(n^{20})$.
  \item The algorithm looks like:\newline
  \begin{verbatim}
  min_cost={}
  def 2b(i):
      if i==n:
          return 0
      if min_cost[i] already exists:
          return min_cost[i]
      else:
          answer=MAX_INT
          for j in range(i+1,n+1):
              answer=min(answer,c(i,j)+2b(j))
          min_cost[i]=answer
          return answer
   \end{verbatim}
   This algorithm has a time complexity of $O(n^2)$ with dynamic programming.
   \item If we draw out the rectangle and fill out each block with the number of ways to get there from the upper-left, then it is quite obvious that there are $\frac{(2n-2)!}{((n-1)!)^2}$ ways.
   \item Since $\mathbf w$ is independent of the summations, just factor it out. We get:\newline
   $f(\mathbf w) = \mathbf w^2 \sum_{i=1}^n \sum_{j=1}^n (a_i^\top - b_j^\top)^2 + \lambda \mathbf \sum_{i=1}^d \mathbf{w}_i$\newline
   In preprocessing we can calculate $\\sum_{i=1}^n \sum_{j=1}^n (a_i^\top - b_j^\top)^2$ first, which takes $O(nd^2)$ time. Then for any given $\mathbf{w}$ we can simply compute $\sum_{i=1}^d \mathbf{w}_i$, which takes $O(d^2)$ time.
\end{enumerate}

\end{document}